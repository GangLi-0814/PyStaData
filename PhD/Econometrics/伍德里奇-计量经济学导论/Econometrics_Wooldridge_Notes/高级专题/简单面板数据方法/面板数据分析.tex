\documentclass[UTF8]{ctexbeamer}

\usetheme{Pittsburgh}
\usecolortheme{whale} 

\usepackage{graphicx}
\usepackage{animate}
\usepackage{color}
\usepackage{hyperref}
\usepackage{hyperref}
\hypersetup{
    colorlinks=true,
    linkcolor=blue,
    filecolor=blue,
    urlcolor=blue,
    citecolor=cyan,
}
\usepackage{tikz}
\usetikzlibrary{decorations.pathreplacing}
\usepackage{amsmath} % 公式
\usepackage{booktabs} %表格框线
\usepackage{xcolor}
\usepackage{multirow}
\usepackage{setspace} % 调整行距
\setbeamertemplate{caption}[numbered] % 图形编号
\usepackage{threeparttable} % 表格添加注释
\usepackage{multicol} % 目录分栏

 % 设置字体
\usepackage{xeCJK}
\setCJKmainfont[BoldFont=STSongti-SC-Regular, ItalicFont=STSongti-SC-Regular]{STSongti-SC-Regular}
\setCJKsansfont[BoldFont=STSongti-SC-Regular]{STSongti-SC-Regular}
\setCJKmonofont{STSongti-SC-Regular}
\setmainfont{TimesNewRomanPSMT}  %英文字体


%%%%%%%%%%%%
% 封面
%%%%%%%%%%%%
\begin{document}
%\logo{\includegraphics[height=0.12\textwidth]{figures/logo_zuel.jpg}}
\title{面板数据分析}
\author{公众号:PyStaData}
%\institute{}
% \institute{gang.li.0814@gmail.com}
\date{\today}

\frame{\titlepage} %生成标题页

%%%%%%%%%%%%
% 目录
%%%%%%%%%%%%

\begin{frame}{目录}
\begin{multicols}{2}
  \tableofcontents
\end{multicols}
\end{frame}

%%%%%%%%%%%%
% 多元回归分析:估计
%%%%%%%%%%%%

%%%%%%%%%%%%%%%
% 1. 混合横截面数据
%%%%%%%%%%%%%%%
\section{混合横截面数据}
\subsection{混合横截面数据}
\begin{frame}{混合横截面数据}
\end{frame}


%%%%%%%%%%%%%%%
% 2.  两时期面板数据分析
%%%%%%%%%%%%%%%
\section{两时期面板数据分析}
\subsection{两时期面板数据分析}
\begin{frame}{两时期面板数据分析}
\end{frame}

%%%%%%%%%%%%%%%
% 3.  多于两期的差分法
%%%%%%%%%%%%%%%
\section{多于两期的差分法}
\subsection{多于两期的差分法}
\begin{frame}{多于两期的差分法}
\end{frame}


%%%%%%%%%%%%%%%%
% 4. 固定效应
%%%%%%%%%%%%%%%%
\section{固定效应}
\subsection{虚拟变量回归}
\begin{frame}{虚拟变量回归}
\end{frame}

\subsection{固定效应还是一阶差分?}
\begin{frame}{固定效应还是一阶差分?}
\end{frame}

\subsection{非平衡面板数据的固定效应法}
\begin{frame}{非平衡面板数据的固定效应法}
\end{frame}



%%%%%%%%%%%%%%%%
% 5. 随机效应
%%%%%%%%%%%%%%%%
\section{随机效应}
\subsection{随机效应还是固定效应?}
\begin{frame}{随机效应还是固定效应?}
\end{frame}


%%%%%%%%%%%%%%%
% 结语
%%%%%%%%%%%%%%%
\begin{frame}{结束语}
\linespread{1.25}
\begin{center}
%{\huge \emph{\textcolor{blue}{感谢聆听,欢迎批评指正!}}}\\
\vspace{5mm}\large
\begin{tabular}{ll}
{\sc 姓名}:  & \textsf{李 刚}\\
{\sc 单位}: & 中南财经政法大学工商管理学院 \\
{\sc 地址}: & 湖北省武汉市洪山区南湖大道182号 \\
{\sc 邮箱}: & \href{mailto:gang.li@stu.zuel.edu.cn}{\color{blue}gang.li@stu.zuel.edu.cn}\\
\end{tabular}
\end{center}
\end{frame}

\end{document} 