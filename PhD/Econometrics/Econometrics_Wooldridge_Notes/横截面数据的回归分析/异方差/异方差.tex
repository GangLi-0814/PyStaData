\documentclass[UTF8]{ctexbeamer}

\usetheme{Pittsburgh}
\usecolortheme{whale} 

\usepackage{graphicx}
\usepackage{animate}
\usepackage{color}
\usepackage{hyperref}
\usepackage{hyperref}
\hypersetup{
    colorlinks=true,
    linkcolor=blue,
    filecolor=blue,
    urlcolor=blue,
    citecolor=cyan,
}
\usepackage{tikz}
\usetikzlibrary{decorations.pathreplacing}
\usepackage{amsmath} % 公式
\usepackage{booktabs} %表格框线
\usepackage{xcolor}
\usepackage{multirow}
\usepackage{setspace} % 调整行距
\setbeamertemplate{caption}[numbered] % 图形编号
\usepackage{threeparttable} % 表格添加注释
\usepackage{multicol} % 目录分栏

 % 设置字体
\usepackage{xeCJK}
\setCJKmainfont[BoldFont=STSongti-SC-Regular, ItalicFont=STSongti-SC-Regular]{STSongti-SC-Regular}
\setCJKsansfont[BoldFont=STSongti-SC-Regular]{STSongti-SC-Regular}
\setCJKmonofont{STSongti-SC-Regular}
\setmainfont{TimesNewRomanPSMT}  %英文字体


%%%%%%%%%%%%
% 封面
%%%%%%%%%%%%
\begin{document}
%\logo{\includegraphics[height=0.12\textwidth]{figures/logo_zuel.jpg}}
\title{违背经典假设模型(1):异方差}
\author{公众号:PyStaData}
%\institute{}
% \institute{gang.li.0814@gmail.com}
\date{\today}

\frame{\titlepage} %生成标题页

%%%%%%%%%%%%
% 目录
%%%%%%%%%%%%

\begin{frame}{目录}
\begin{multicols}{2}
  \tableofcontents
\end{multicols}
\end{frame}

%%%%%%%%%%%%
% 导论
%%%%%%%%%%%%


%%%%%%%%%%%%%%%
% 1. 定义及后果 
%%%%%%%%%%%%%%%
\section{异方差的定义及后果 }
\subsection{异方差的定义}
\begin{frame}{定义}
\linespread{1.5}
“条件异方差”(简称“异方差”)是违背球型扰动假设的一种情形,即条件方差 $Var(\epsilon_i|X)$ 依赖于 $i$ ( 为 $\sigma _i ^2$ ),而不是常数 $\sigma^2$ 。
\end{frame}

\subsection{异方差的后果}
\begin{frame}{异方差的后果}
\linespread{1.5}
\begin{enumerate}
  \item $\beta$ 估计量无偏:用 OLS估计所得参数估计量 $\hat \beta$ 仍具有无偏性,即 $E(\hat \beta) = \beta$ 。
  \item $\beta$ 估计量非有效:存在异方差时,$\hat \beta_{OLS}$ 不是 $\beta$ 的有效估计;直接计算 $Se(\hat \beta)$ 有误 。
  \item t检验、F 检验失效
\end{enumerate}
\end{frame}

%%%%%%%%%%%%%%%
% 2. 异方差检验
%%%%%%%%%%%%%%%
\section{异方差的检验 }
\subsection{图示法}
\begin{frame}{图示法}
\end{frame}

\subsection{ BP 检验}
\begin{frame}{ BP 检验}
\end{frame}

\subsection{ White 检验}
\begin{frame}{ White 检验}
\end{frame}

%%%%%%%%%%%%%%%
% 3. 异方差的处理
%%%%%%%%%%%%%%%
\section{异方差的处理}
\subsection{OLS + 稳健标准误}
\begin{frame}{OLS + 稳健标准误}
\end{frame}

\subsection{加权最小二乘回归(WLS)}
\begin{frame}{加权最小二乘回归(WLS)}
\end{frame}

\subsection{可行加权最小二乘回归(FWLS)}
\begin{frame}{可行加权最小二乘回归(FWLS)}
\end{frame}

%%%%%%%%%%%%%%%
% 4. Stata 命令及实例
%%%%%%%%%%%%%%%
\section{ Stata 命令及实例}
\subsection{命令介绍}
\begin{frame}{命令介绍}
\end{frame}

\subsection{案例分析}
\begin{frame}{案例分析}
\end{frame}

%%%%%%%%%%%%%%%
% 结语
%%%%%%%%%%%%%%%
\begin{frame}{结束语}
\linespread{1.25}
\begin{center}
%{\huge \emph{\textcolor{blue}{感谢聆听,欢迎批评指正!}}}\\
\vspace{5mm}\large
\begin{tabular}{ll}
{\sc 姓名}:  & \textsf{李 刚}\\
{\sc 单位}: & 中南财经政法大学工商管理学院 \\
{\sc 地址}: & 湖北省武汉市洪山区南湖大道182号 \\
{\sc 邮箱}: & \href{mailto:gang.li@stu.zuel.edu.cn}{\color{blue}gang.li@stu.zuel.edu.cn}\\
\end{tabular}
\end{center}
\end{frame}

\end{document} 