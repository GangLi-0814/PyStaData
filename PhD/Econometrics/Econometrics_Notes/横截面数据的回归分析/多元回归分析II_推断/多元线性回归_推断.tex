\documentclass[UTF8]{ctexbeamer}

\usetheme{Pittsburgh}
\usecolortheme{whale} 

\usepackage{graphicx}
\usepackage{animate}
\usepackage{color}
\usepackage{hyperref}
\usepackage{hyperref}
\hypersetup{
    colorlinks=true,
    linkcolor=blue,
    filecolor=blue,
    urlcolor=blue,
    citecolor=cyan,
}
\usepackage{tikz}
\usetikzlibrary{decorations.pathreplacing}
\usepackage{amsmath} % 公式
\usepackage{booktabs} %表格框线
\usepackage{xcolor}
\usepackage{multirow}
\usepackage{setspace} % 调整行距
\setbeamertemplate{caption}[numbered] % 图形编号
\usepackage{threeparttable} % 表格添加注释
\usepackage{multicol} % 目录分栏

 % 设置字体
\usepackage{xeCJK}
\setCJKmainfont[BoldFont=STSongti-SC-Regular, ItalicFont=STSongti-SC-Regular]{STSongti-SC-Regular}
\setCJKsansfont[BoldFont=STSongti-SC-Regular]{STSongti-SC-Regular}
\setCJKmonofont{STSongti-SC-Regular}
\setmainfont{TimesNewRomanPSMT}  %英文字体


%%%%%%%%%%%%
% 封面
%%%%%%%%%%%%
\begin{document}
%\logo{\includegraphics[height=0.12\textwidth]{figures/logo_zuel.jpg}}
\title{多元回归分析(2):推断}
\author{公众号:PyStaData}
%\institute{}
% \institute{gang.li.0814@gmail.com}
\date{\today}

\frame{\titlepage} %生成标题页

%%%%%%%%%%%%
% 目录
%%%%%%%%%%%%

\begin{frame}{目录}
\begin{multicols}{2}
  \tableofcontents
\end{multicols}
\end{frame}

%%%%%%%%%%%%
% 多元回归分析:估计
%%%%%%%%%%%%

%%%%%%%%%%%%%%%
% 1. OLS 估计量的抽样分布
%%%%%%%%%%%%%%%
\section{OLS 估计量的抽样分布}
\subsection{OLS 估计量的抽样分布}
\begin{frame}{OLS 估计量的抽样分布}
\end{frame}


%%%%%%%%%%%%%%%
% 2.  t 检验
%%%%%%%%%%%%%%%
\section{t 检验}
\subsection{t 检验}
\begin{frame}{t 检验}
\end{frame}

%%%%%%%%%%%%%%%
% 3.  p 值
%%%%%%%%%%%%%%%
\section{p 值}
\subsection{p 值}
\begin{frame}{p 值}
\end{frame}



%%%%%%%%%%%%%%%
% 3.  置信区间
%%%%%%%%%%%%%%%
\section{置信区间}
\subsection{置信区间}
\begin{frame}{置信区间}
\end{frame}

%%%%%%%%%%%%%%%
% 4.  F 检验
%%%%%%%%%%%%%%%
\section{F 检验}
\subsection{F 检验}
\begin{frame}{F 检验}
\end{frame}

%%%%%%%%%%%%%%%%
% 5. 经典线性模型假定
%%%%%%%%%%%%%%%%
\section{经典线性模型假定}
\subsection{经典线性模型假定}
\begin{frame}{经典线性模型假定}
\end{frame}

%%%%%%%%%%%%%%%
% 结语
%%%%%%%%%%%%%%%
\begin{frame}{结束语}
\linespread{1.25}
\begin{center}
%{\huge \emph{\textcolor{blue}{感谢聆听,欢迎批评指正!}}}\\
\vspace{5mm}\large
\begin{tabular}{ll}
{\sc 姓名}:  & \textsf{李 刚}\\
{\sc 单位}: & 中南财经政法大学工商管理学院 \\
{\sc 地址}: & 湖北省武汉市洪山区南湖大道182号 \\
{\sc 邮箱}: & \href{mailto:gang.li@stu.zuel.edu.cn}{\color{blue}gang.li@stu.zuel.edu.cn}\\
\end{tabular}
\end{center}
\end{frame}

\end{document} 