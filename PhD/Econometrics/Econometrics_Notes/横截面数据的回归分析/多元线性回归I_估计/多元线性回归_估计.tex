\documentclass[UTF8]{ctexbeamer}

\usetheme{Pittsburgh}
\usecolortheme{whale} 

\usepackage{graphicx}
\usepackage{animate}
\usepackage{color}
\usepackage{hyperref}
\usepackage{hyperref}
\hypersetup{
    colorlinks=true,
    linkcolor=blue,
    filecolor=blue,
    urlcolor=blue,
    citecolor=cyan,
}
\usepackage{tikz}
\usetikzlibrary{decorations.pathreplacing}
\usepackage{amsmath} % 公式
\usepackage{booktabs} %表格框线
\usepackage{xcolor}
\usepackage{multirow}
\usepackage{setspace} % 调整行距
\setbeamertemplate{caption}[numbered] % 图形编号
\usepackage{threeparttable} % 表格添加注释
\usepackage{multicol} % 目录分栏

 % 设置字体
\usepackage{xeCJK}
\setCJKmainfont[BoldFont=STSongti-SC-Regular, ItalicFont=STSongti-SC-Regular]{STSongti-SC-Regular}
\setCJKsansfont[BoldFont=STSongti-SC-Regular]{STSongti-SC-Regular}
\setCJKmonofont{STSongti-SC-Regular}
\setmainfont{TimesNewRomanPSMT}  %英文字体


%%%%%%%%%%%%
% 封面
%%%%%%%%%%%%
\begin{document}
%\logo{\includegraphics[height=0.12\textwidth]{figures/logo_zuel.jpg}}
\title{多元回归分析(1):估计}
\author{公众号:PyStaData}
%\institute{}
% \institute{gang.li.0814@gmail.com}
\date{\today}

\frame{\titlepage} %生成标题页

%%%%%%%%%%%%
% 目录
%%%%%%%%%%%%

\begin{frame}{目录}
\begin{multicols}{2}
  \tableofcontents
\end{multicols}
\end{frame}

%%%%%%%%%%%%
% 多元回归分析:估计
%%%%%%%%%%%%

%%%%%%%%%%%%%%%
% 1. OLS 推导
%%%%%%%%%%%%%%%
\section{OLS 推导 }
\subsection{OLS 推导}
\begin{frame}{OLS 推导}
\end{frame}


%%%%%%%%%%%%%%%
% 2. 拟合优度
%%%%%%%%%%%%%%%
\section{拟合优度}
\subsection{拟合优度}
\begin{frame}{拟合优度}
\end{frame}

%%%%%%%%%%%%%%%%%%%%
% 3. OLS 估计量的期望值
%%%%%%%%%%%%%%%%%%%%
\section{OLS 估计量的期望值}
\subsection{OLS 的无偏性}
\begin{frame}{OLS 的无偏性}
\end{frame}

\subsection{包含无关变量}
\begin{frame}{包含无关变量}
\end{frame}

\subsection{遗漏变量偏误}
\begin{frame}{遗漏变量偏误}
\end{frame}


%%%%%%%%%%%%%%%%%%%%
% 3. OLS 估计量的方差
%%%%%%%%%%%%%%%%%%%%
\section{OLS 估计量的方差}
\subsection{同方差性假定}
\begin{frame}{同方差性假定}
\end{frame}

\subsection{OLS 方差的成分:多重共线性}
\begin{frame}{OLS 方差的成分:多重共线性}
\end{frame}

\subsection{估计 $\sigma^2$ :OLS 估计量的标准误}
\begin{frame}{估计 $\sigma^2$ :OLS 估计量的标准误}
\end{frame}

%%%%%%%%%%%%%%%%%%%%%%%%%
% 4. OLS 估计的有效性:高斯-马尔科夫定理
%%%%%%%%%%%%%%%%%%%%%%%%%
\section{OLS 估计的有效性:高斯-马尔科夫定理}
\subsection{OLS 估计的有效性:高斯-马尔科夫定理}
\begin{frame}{OLS 估计的有效性:高斯-马尔科夫定理}
\end{frame}

%%%%%%%%%%%%%%%
% 结语
%%%%%%%%%%%%%%%
\begin{frame}{结束语}
\linespread{1.25}
\begin{center}
%{\huge \emph{\textcolor{blue}{感谢聆听,欢迎批评指正!}}}\\
\vspace{5mm}\large
\begin{tabular}{ll}
{\sc 姓名}:  & \textsf{李 刚}\\
{\sc 单位}: & 中南财经政法大学工商管理学院 \\
{\sc 地址}: & 湖北省武汉市洪山区南湖大道182号 \\
{\sc 邮箱}: & \href{mailto:gang.li@stu.zuel.edu.cn}{\color{blue}gang.li@stu.zuel.edu.cn}\\
\end{tabular}
\end{center}
\end{frame}

\end{document} 